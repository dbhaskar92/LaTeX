% LaTeX presentation for CUMC 2014
% Author: Dhananjay Bhaskar <dbhaskar92@gmail.com>

\documentclass{beamer}

\usetheme{Rochester}
\usecolortheme{dolphin}
\usepackage{graphicx}
\usepackage{multicol}
\usepackage{amsfonts} 

\usepackage{media9}
\newcommand{\includemovie}[3]{
	\includemedia[
		width=#1,height=#2,
		activate=pagevisible,
		deactivate=pageclose,
		addresource=#3,
		flashvars={
			src=#3			% same path as in addresource!
			&autoPlay=false	% if = true, automatically starts playback after activation
			&loop=false		% if loop=true, media is played in a loop
			&controlBarAutoHideTimeout=1	% time span before auto-hide
		}
	]{}{StrobeMediaPlayback.swf}
}

\setbeamertemplate{navigation symbols}{}	% remove navigation symbols
\setbeamertemplate{footline}[page number]	% add page numbers

\begin{document}

\title{Mathematical Models of Epithelial Monolayer}
   
\author[Bhaskar,Keshet]{%
  Dhananjay Bhaskar\inst{1} \and
  Dr. Leah Keshet (Supervisor)\inst{2}}

\institute{
  \inst{1}
  Combined Major in Computer Science and Mathematics\\
  University of British Columbia
  \and
  \inst{2}
  Department of Mathematics\\
  University of British Columbia}
  
\date[CUMC 2014]{Canadian Undergraduate Mathematics Conference\\
	Carleton University, Ottawa, 2014}

\frame{\titlepage} 

\frame{\frametitle{Outline}\tableofcontents}

\section{Multiscale Modeling and CHASTE} 

\subsection{Biological Background and Motivation}

\frame{\frametitle{Biological Background and Motivation}
\begin{minipage}{0.5\linewidth}  
\begin{itemize}
\item \textbf{Epithelium}: layers of cells that line hollow organs and glands in animal tissue\\
\item Epithelial cells are arranged in single or multiple layers, depending on the organ and location\\
\item We are interested in modeling \textbf{epithelial monolayers} to simulate biological processes like cell migration, monoclonal conversion, wound healing, etc.
\end{itemize}
\end{minipage}
\hspace{5mm}
\begin{minipage}{0.4\linewidth}  
	\begin{figure}[h!]
	\centering
	\includemovie{150pt}{120pt}{videos/migratinghumankidney.mp4}
	\caption{Migrating human kidney epithelial cells}
	\end{figure}
\end{minipage}
} 

\frame{\frametitle{Multiscale Modeling in Biology}
\begin{figure}[h!]
	\centering
	\includegraphics[scale=0.5]{images/scales}
	\caption{Multiple possible models at each scale}
\end{figure}
\begin{itemize}
\item Processes on each biological scale depend on other scales\\
\item Cells interact with each other and the environment to influence tissue level\\
\item Models must incorporate features from multiple scales
\end{itemize}
}

\subsection{Cancer, Heart and Soft Tissue Environment}

\frame{\frametitle{Cancer, Heart and Soft Tissue Environment}
\begin{minipage}{0.7\linewidth}   
	\begin{itemize}
	\item Extensible, general purpose simulation framework
	\item Development team based in the Computational Biology Group at University of Oxford
	\item Written in C++ using Agile methodology
	\item Useful for solving ODE, PDE and linear systems arising from biological and physiological models
	\item Research focus on cardiac electrophysiology, systems biology and cell-based modeling (soft tissue mechanics)
	\end{itemize}
\end{minipage}
\begin{minipage}{0.2\linewidth}
	\includegraphics[scale=0.4]{images/ChasteLogo}      
\end{minipage}
}


\frame{\frametitle{CHASTE Simulation Structure}
\begin{minipage}{0.5\linewidth}
	\begin{itemize}
	\item \textbf{Subcellular level:} cell division, cell death, cell cycle models \\ \vspace{.1cm}
	\item \textbf{Cell level:} different representations of cell, inter-cellular forces \\ \vspace{.1cm}
	\item \textbf{Tissue level:} external factors, reaction diffusion PDEs
	\end{itemize}
\end{minipage}
\begin{minipage}{0.3\linewidth}
	\includegraphics[scale=0.3]{images/chastesim}  
\end{minipage}  
}


\section{Modeling Individual Cells}

\frame{\frametitle{Modeling Individual Cells}
\begin{exampleblock}{On-lattice models}
Cells occupy discretized positions in a lattice (grid). Examples:
	\begin{itemize}
	\item Cellular Automata
	\item Cellular Potts
	\end{itemize}
\end{exampleblock}

\begin{exampleblock}{Off-lattice models}
Cells can move freely in space in according to interactions with neighbouring cells. Examples:
	\begin{itemize}
	\item Cell centre (node-based) model
	\item Vertex dynamics model
	\end{itemize}
\end{exampleblock}
}

\subsection{Cellular Automata}
\frame{\frametitle{Cellular Automata}
\begin{minipage}{0.7\linewidth}
	\begin{itemize}
	\item Fixed lattice, cells divide into neighbouring lattice sites\\ \vspace{0.2cm}
	\item Simple implementation, computationally inexpensive\\ \vspace{0.2cm}
	\item \textbf{Disadvantages:} 
		\begin{itemize}
		\item Cells have fixed size, discrete positions\\ \vspace{0.1cm}
		\item Anistropic evolution depending on initial configuration
		\end{itemize}
	\end{itemize}
\end{minipage}
\begin{minipage}{0.2\linewidth}
	\includegraphics[scale=0.5]{images/automata}  
\end{minipage}  
}

\subsection{Cellular Potts Model}
\frame{\frametitle{Cellular Potts Model}
\begin{minipage}{0.7\linewidth}
	\begin{itemize}
	\item Cells are composed of a collection of lattice sites\\ \vspace{0.1cm}
	\item Additional sites are included or removed to minimize an energy function\\ \vspace{0.1cm}
	\item Monte Carlo simulations to update cells\\ \vspace{0.1cm}
	\item Flexible representation of cell shape\\ \vspace{0.1cm}
	\item \textbf{Disadvantages:} 
		\begin{itemize}
		\item Difficulty in capturing viscuous deformations\\ \vspace{0.1cm}
		\item Additional constraints required to prevent cells sliding past each other
		\end{itemize} 
	\end{itemize}
\end{minipage}
\begin{minipage}{0.2\linewidth}
	\includegraphics[scale=0.25]{images/potts}  
\end{minipage}  
}

\frame{\frametitle{Cellular Potts Simulation - Demo}
\begin{figure}[h!]
  	\centering
  	\includemovie{200pt}{200pt}{videos/pottsbasedmonolayer.mp4}
\end{figure}
}

\frame{\frametitle{CPM - Mathematical Notation}
$S$ - set of lattice sites\\
$W$ - set of cell indices, $W = \{0,1,...,n\} \quad n \in \mathbb{N}$\\
Let $\tau : W \to \Lambda$ be a mapping of cell indices to cell type.\\
CPM assigns value $\eta(x) \in W \quad \forall x \in S$\\ \vspace{0.1cm}
\begin{center}
	\includegraphics[scale=4.0]{images/cpm}  
\end{center}
Given a configuration $\eta$, a cell in CPM is set of all lattice sites with the same cell index.\\
Cell $w := \{x \in S : \eta(x)=w\} \quad w \in W$ \textbackslash $\{0\}$
}

\frame{\frametitle{CPM - Cell Volume and Surface Area}
Cell with index $w$ has volume (area):\\
\begin{equation}
V_{w}(\eta) = \sum_{x \in S} \delta(\eta(x),w)
\end{equation}
Surface area (perimeter) of cell with index $w$:\\
\begin{equation}
S_{w}(\eta) = 1/2 \sum_{\text{interfaces$\{x,y\}$}} \delta(\eta(x),w)
\end{equation}
Cellular Potts simulation is a Markov chain over the set of possible configurations $\eta$ where transition probabilities
are specified with the help of a hamiltonian (energy).
}

\frame{\frametitle{Typical Structure of CPM Hamiltonian}
\begin{equation}
H = H_{V} + H_{I} + H_{0}
\end{equation}

Volume constrain term:
\begin{equation}
H_{V}(\eta) = \sum_{w \in W} \lambda_{\tau(w)} [V_{w}(\eta) - v_{\tau(w)}]^2
\end{equation}

Surface interaction/adhesion term:
\begin{equation}
H_{I}(\eta) = \sum_{\text{interfaces$\{x,y\}$}} J(\tau(\eta(x)),\tau(\eta(y)))
\end{equation}

$J : \Lambda \times \Lambda \to \mathbb{R}$ is a symmetric matrix that specifies cell-cell adhesion between all cell types and the medium
}

\frame{\frametitle{Cellular Potts Simulation}
\begin{block}{Metropolis Algorithm}
\begin{enumerate}
	\item Start with initial configuration $\eta$
	\item Pick a target site $x \in S$ and a neighbour $y$ of $x$ uniformly
	\item Calculate energy gain $\Delta H_{x}^{y} = H(\eta_{x}^{y}) - H(\eta)$
	\item If $\Delta H_{x}^{y} < h$, accept new configuration
	\item Otherwise, accept with probability:
	\begin{equation}
		P(\eta = \eta_{x}^{y}) = e^{-(\Delta H_{x}^{y} - h)/T}
	\end{equation}
\end{enumerate}
\end{block}
}

\subsection{Cell Centre Model}
\frame{\frametitle{Cell Centre Model}
\begin{minipage}{0.7\linewidth}
	\begin{itemize}
	\item Represent cells as points in space\\ \vspace{0.1cm}
	\item Forces between centres modelled as overdamped springs:\\
	$\gamma \dfrac{dr_{i}}{dt} = \sum_{j} F_{ij} \dfrac{r_{i} - r_{j}}{\|r_{i} - r_{j}\|}$\\
	\item Cell connectivity:\\
		\begin{itemize}
		\item Overlapping Spheres Model (node-based)
		\item Voronoi Tessellation (mesh-based)
		\end{itemize}
	\item Smooth motion\\ \vspace{0.1cm}
	\item No direct control of cell size and cell-cell interactions
	\end{itemize}
\end{minipage}
\begin{minipage}{0.2\linewidth}
	\includegraphics[scale=0.25]{images/cellcentre}  
\end{minipage}  
}

\frame{\frametitle{Node Based Model - Demo}
\begin{figure}[h!]
  	\centering
  	\includemovie{200pt}{200pt}{videos/nodebasedmonolayer.mp4}
\end{figure}
}

\subsection{Vertex Dynamics Model}
\frame{\frametitle{Vertex Dynamics Model}
\begin{minipage}{0.7\linewidth}
	\begin{itemize}
	\item Cells represented as polygons whose vertices are free to move\\ \vspace{0.1cm}
	\item Vertex motion is overdamped (ignoring inertia):\\
	$\eta_{i} \dfrac{dr_{i}}{dt} = F_{i}(t)$\\
	\item Modelling of force on vertex:
		\begin{itemize}
		\item Weliky and Oster (1990, explicit)
		\item Nagai and Honda (2001, potential-based)
		\end{itemize}
	\item Control over cell size and cell-cell adhesion/interactions
	\item Good model for epithelia that exhibit regular shape
	\end{itemize}
\end{minipage}
\begin{minipage}{0.2\linewidth}
	\includegraphics[scale=0.25]{images/vertexdynamics}  
\end{minipage}  
}

\frame{\frametitle{Weliky and Oster Model}
Let $m_{i}(t)$ be the number of elements containing vertex $i$ at time $t$, then the total force acting on the vertex, 
$F_{i}(t)$ is given by

\begin{equation}
\mathbf{F_{i}} = \sum_{k=1}^{m_{i}} \mathbf{f_{k}} = \sum_{k=1}^{m_{i}} [\mathbf{p_{i}^{k}} + \mathbf{t_{il}^{k}} + \mathbf{t_{ir}^{k}}]
\end{equation}
\begin{center}
\includegraphics[scale=0.25]{images/weliky}
\end{center}

Where cytocortical pressure $\mathbf{p_{i}^{k}} = \dfrac{\rho \mathbf{\hat{p_{i}^{k}}}}{A_{k}}$\\
and membrane tension $\mathbf{t_{i}^{k}} = \kappa C_{k} \mathbf{\hat{t_{i}^{k}}}$
}

\frame{\frametitle{Nagai and Honda Model}
Model potential taking into account the cell's limited ability to undergo deformations, volumetric changes, etc. due to adhesion to 
other cells. The authors define free energy (work function) of the system and use it to derive force acting on each vertex.
\begin{equation}
U = U_{D} + U_{S} + U_{A} = \sum_{k=1}^{N} (U_{D}^{k} + U_{S}^{k} + U_{A}^{k}) 
\end{equation}

The deformation energy term $U_{D}^{k}$ ensures each cell attains its target volume:
$U_{D}^{k} = \lambda (A_{k} - A_{0_{K}})^2$\\ \vspace{0.2cm}

Assume cell $k$ has $n_{k}$ vertices $ \{(x_{i},y_{j})\}_{j=0}^{n_{k}-1}$, then\\
$A_{k} = 1/2 |\sum_{j=0}^{n_{k}-1} (x_{j}y_{j+1} - x_{j+1}y_{j}) |$
}

\frame{\frametitle{Nagai and Honda Model}
The membrane surface energy $U_{S}^{k}$ aims to conserve cell membrane length and drives each cell $k$ to a circular shape.\\ 
\begin{center}
$U_{S}^{k} = \beta (C_{k} - C_{0_{K}})^2$
\end{center}
The target perimeter $C_{0_{K}}$ is the circumference of a circular cell of area $A_{0_{K}}$: $C_{0_{K}} = 2 \sqrt{\pi A_{0_{K}}}$ \\
\vspace{0.2cm}
The cell-cell adhesion energy $U_{A}^{k}$ represents the free energy associated with bonds between each cell $k$ and its neighbours:\\
\begin{center}
$U_{A}^{k} = \sum_{j=0}^{n_{k}-1} \gamma_{k,j} d_{k,j}$
\end{center}
Where $\gamma_{k,j}$ is a positive constant dependent on the cell types in contact and $d_{k,j}$ is the distance between the $j^{th}$ vertex of cell $k$ and the next in an anticlockwise direction.
}

\frame{\frametitle{Nagai and Honda Model}
The gradient of free energy is assumed to exert a force on each vertex of the monolayer:\\
$F_{i} = - \nabla_{i} U = - \nabla_{i} \sum_{k \in \mathcal{N}_{i}} (U_{D}^{k} + U_{S}^{k} + U_{A}^{k})$\\ \vspace{0.2cm}

\begin{block}{Numerical Implementation}
\begin{enumerate}
	\item Update cell properties from any subcellular model
	\item Implement mesh restructuring operations
	\item Loop over all cells, calculate force $F_{i}$ for each vertex $i$
	\item Update the vertex positions simultaneously using Forward Euler scheme:
\begin{equation}
\mathbf{r_{i}}(t + \Delta t) = \mathbf{r_{i}}(t) + \dfrac{\Delta t}{\eta_{i}} \mathbf{F_{i}}(t)
\end{equation}
\end{enumerate}
\end{block}
}

\frame{\frametitle{Vertex Dynamics Model - Demo}
\begin{figure}[h!]
  	\centering
  	\includemovie{200pt}{200pt}{videos/vertexbasedmonolayer.mp4}
\end{figure}
}

\frame{\frametitle{Variations of Vertex Dynamics Model}
\begin{itemize}
\item Active cell migration:
\begin{equation}
\eta_{i} \dfrac{d\mathbf{r_{i}}}{dt} = \mathbf{F_{i}}(t) + \alpha \hat{e_{y}}
\end{equation}
\vspace{0.2cm}
\item Random motion of vertices:
\begin{equation}
\eta_{i} \dfrac{d\mathbf{r_{i}}}{dt} = \mathbf{F_{i}} + \mathbf{F_{i}^{\text{random}}}
\end{equation}
\end{itemize}
}


%\section{Tissue and Sub-cellular Models} 

%\subsection{Subcellular Processes - Cell Cycle Models}
%\frame{\frametitle{Subcellular Processes - Cell Cycle Models}
%\begin{itemize}
%\item Subcellular processes determine how cells grow, divide and die\\
%\vspace{0.2cm}
%\item Simple rule-based models
%	\begin{itemize}
%	\item Deterministic Models: \textit{FixedDurationCellCycleModel}\\ \vspace{0.1cm}
%	\item Stochastic Models: \textit{StochasticCellCycleModel}, \textit{StochasticAreaDependentCellCycleModel}
%	\end{itemize}
%\vspace{0.2cm}
%\item Model cell cycle and other metabolic pathways
%	\begin{itemize}
%	\item System of non-linear ODEs\\ \vspace{0.1cm}
%	\item Coupled to extracellular concentrations of nutrients, inhibitors, signalling proteins etc.\\ \vspace{0.1cm}
%	\item Determines cell division, cell-cell adhesion, cell size/shape, etc.
%	\end{itemize}
%\end{itemize}
%}

\subsection{Modeling Tissue-Level Processes}
\frame{\frametitle{Modeling Tissue-Level Processes}
\begin{minipage}{0.5\linewidth}
\begin{itemize}
\item Geometric constraints (boundary conditions)\\ \vspace{0.2cm}
\item Imposed gradients\\ \vspace{0.2cm}
\item Vascular networks (obstructions)\\ \vspace{0.2cm}
\item Field equations:
	\begin{itemize}
	\item Reaction-diffusion PDEs\\ \vspace{0.1cm}
	\item Cells act as sinks/sources
	\end{itemize}
\end{itemize}
\end{minipage}
\hspace{0.2cm}
\begin{minipage}{0.4\linewidth}
	\begin{figure}[h!]
  	\centering
    \includegraphics[scale=0.2]{images/crypt}
    \caption{Gradient of Wnt signalling factors across crypt axis}
	\end{figure} 
\end{minipage}  
}

\frame{\frametitle{Cell Killers and Tissue Boundary}
\begin{figure}[h!]
  	\centering
  	\includemovie{200pt}{200pt}{videos/vertexperiodicwithkiller.mp4}
\end{figure}
}


\section{Exploring Tissue Dynamics}

\subsection{Contact Inhibition}
\frame{\frametitle{Contact Inhibition}
\begin{minipage}[t]{0.5\linewidth}  
\begin{flushleft}
	\begin{figure}[h!]
  		\centering
  		\includemovie{150pt}{150pt}{videos/nodecontactinhibitbox.mp4}
	\end{figure}
\end{flushleft}
\end{minipage}
\begin{minipage}[t]{0.4\linewidth}
\begin{flushright}
	\begin{figure}[h!]
  	\centering
  	\includemovie{150pt}{150pt}{videos/vertexcontactinhibitbox.mp4}
\end{figure}
\end{flushright}
\end{minipage}
}

\subsection{Cell Sorting}
\frame{\frametitle{Cell Sorting}
\begin{figure}[h!]
  	\centering
  	\includemovie{200pt}{200pt}{videos/vertexcellsorting.mp4}
\end{figure}
}

\subsection{Monoclonal Conversion}
\frame{\frametitle{Monoclonal Conversion}
\begin{figure}[h!]
  	\centering
  	\includemovie{200pt}{200pt}{videos/nodemonoclonalconversion.mp4}
\end{figure}
}

\frame{\frametitle{Current Work}
I am working on the following this summer:\\ \vspace{0.2cm}
\begin{itemize}
\item Develop simulations of vertex-based models to demonstrate cell sorting, wound healing, cell migration, etc.
\item Compare results with Cellular Potts Model and Overlapping Spheres Model
\item Determine how changing various parameters in the model affects the simulation 
\end{itemize}
}

\frame{\frametitle{Further Reading}
\begin{itemize}
\item Fletcher, A. G., Osborne, J. M., et al. "Implementing vertex dynamics models of cell population in biology within a consistent computational framework", Progress in Biophysics and Molecular Biology. 113:299-326. 2013.
\item Nagai, T., and Honda, H. "A dynamic cell model for the formation of epithelial tissues", Philosophical Magazine Part B. 81:699-719
\item Pathmanathan, P., et al. "A computational study of discrete mechanical tissue models", Physical Biology. Vol. 6. No. 3. 2009
\item Voss-Bohme, A. "Multi-Scale Modeling in Morphogenesis: A Critical Analysis of the Cellular Potts Model, PLoS ONE. 7(9): e42852.
\end{itemize}
}

\end{document}
