\section{Empirical Analysis and Results}

After configuring the target algorithm, empirical analysis is useful for determining improvement in performance achieved. In this section, we summarize the results of empirical study comparing the performance of MSPO using the default configuration and configurations obtained from parallel ParamILS runs.\\

As shown in Figure 1, none of configurations obtained from ParamILS dominates over the default configuration. These runlength scatter plots are generated by computing runlength for 10 independent MPSO runs for each of the four configurations using cutoff length of 400000. Timeouts are depicted by points that appear at the edge of the bounding box. Note the large number of timeouts for the shifted, rotated, high conditioned elliptic function.\\
	
\begin{center}
\hfill \includegraphics[scale=0.45]{scatterWeierstrass.png} \quad \includegraphics[scale=0.45]{scatterRosenbrock.png} \hfill \\
\hfill \includegraphics[scale=0.45]{scatterRastrigin.png} \quad \includegraphics[scale=0.45]{scatterElliptic.png} \hfill
\end{center}
\textbf{Figure 1:} Scatter plots of runlength (on logarithmic scale) comparing 10 independent MPSO runs on validation benchmark problems using configured parameter set (y-axis) and default configuration (x-axis). Points corresponding to {\it Run 0}, {\it Run 1} and {\it Run 2} configurations are shown in {\it Red}, {\it Green} and {\it Blue} respectively.\\

For a subset of benchmark problems, namely Ackley{\vtick}s function, shifted rotated Griewank{\vtick}s function, shifted Schwefel{\vtick}s function and shifted sphere function, all configurations derived from ParamILS dominated the default configuration (see Figure 2).\\

Based on exploratory analysis, Run 2 configuration appears to be most promising. Therefore, in order to better understand how this configuration performs compared to the default configuration, we plot a cumulative density function (CDF) for both by computing runlength for 10 independent runs each of 9 benchmark problems taken from validation set. The results are shown in Figure 3. We observe a tradeoff between the default and ParamILS-derived configuration which suggests that we can exploit restart strategies, construct algorithm selectors and portfolios to improve performance further.\\

\begin{center}
\hfill \includegraphics[scale=0.45]{scatterAckley.png} \quad \includegraphics[scale=0.45]{scatterGriewank.png} \hfill \\
\hfill \includegraphics[scale=0.45]{scatterSchwefel.png} \quad \includegraphics[scale=0.45]{scatterSphere.png} \hfill
\end{center}
\textbf{Figure 2:} Cases where the ParamILS configurations dominate the default configuration on 10 independent runs.\\

\begin{center}
\includegraphics[scale=0.5]{CDFPlot.png}
\end{center}	
\textbf{Figure 3:} Runlength distribution of default configuration and Run 2 configuration based on 10 independent runs each for a set of 9 validation benchmark problems.