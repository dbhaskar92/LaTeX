\section{Introduction}

Particle Swarm Optimization (PSO) is a stochastic any-time optimization algorithm that operates on continuous multimodal functions. It is inspired from social behavior observed in flocks of birds and schools of fish (Eberhart et al., 1995). PSO explores the search space of a given problem using a population (called swarm in this context) of search agents (called particles) using information about each particle's previous best performance and the best previous performance of it's neighbors. When one particle locates a candidate target (local optimum of the objective function), it transmits this information to all other particles. Neighboring particles gravitate towards this target using the information they have gathered as well as past memory. PSO has been applied effectively to a wide range of problems in science and engineering, including design of power systems (Abido, 2002), thermal layout optimization (Zang et al., 2012), task assignment problems (Salman et al., 2002), training neural networks (Li and Chen, 2006 and Vilović et al., 2009) and optimizing biochemical processes (Cockshott and Hartman, 2001).\\

Memetic Algorithms (MAs) employ metaheuristics like genetic and evolutionary algorithms to detect promising regions in the search space that might contain the global optimizer and use stochastic local search procedures to probe these regions. MAs can compute the optimum with high accuracy in a large search space. A memetic approach to PSO using Random Walk with Direction Exploitation (RWDE) to perform local search has been shown to be superior for solving unconstrained, constrained, minimax and integer programming problems (Petalas et al., 2007). This approach also works well for optimization problems featuring a discrete search space by rounding real variables to nearest integer values for objective function evaluation (Kennedy and Eberhart, 1997). Although many evolutionary algorithms suffer from search stagnation, PSO has been found to be robust (Laskari et al., 2002).\\

Published experimental data (Petalas et al., 2007) and exploratory analysis of Memetic PSO algorithm (MPSO) based on a MATLAB implementation (result of a summer research project in 2013 for solving the unconstrained binary quadratic programming problem) suggest that the memetic approach outperforms the standard PSO algorithm in terms of finding the global optima. However, a comprehensive empirical analysis of MPSO could not be found while reviewing literature. This report presents the findings of a CPSC 536H course project where we compare performance of MPSO algorithm, using a set of benchmark problems commonly found in literature concerning unconstrained optimization of multimodal functions, before and after parameter tuning and algorithm configuration. We expose all parameters in the memetic PSO algorithm and configure a selection of these parameters to minimize runlength (number of function evaluations) using an existing implementation of ParamILS. We document challenges that emerged during data collection and list promising avenues for future research based on our observations.