\documentclass[a0paper,landscape,final]{baposter}

\usepackage{times}
\usepackage{calc}
\usepackage{graphicx}
\usepackage{amsmath}
\usepackage{amssymb}
\usepackage{relsize}
\usepackage{multirow}
\usepackage{bm}
\usepackage{graphicx}
\usepackage{multicol}
\usepackage{pgfbaselayers}
\usepackage{enumitem}

\pgfdeclarelayer{background}
\pgfdeclarelayer{foreground}
\pgfsetlayers{background,main,foreground}

\usepackage{helvet}
\usepackage{palatino}

\selectcolormodel{cmyk}
\graphicspath{{images/}}

%%%%%%%%%%%%%%%%%%%%%%%%%%%%%%%%%%%%%%%%%%%%%%%%%%%%%%%%%%%%%%%%%%%%%%%%%%%%%%%%
%%%% Some math symbols used in the text
%%%%%%%%%%%%%%%%%%%%%%%%%%%%%%%%%%%%%%%%%%%%%%%%%%%%%%%%%%%%%%%%%%%%%%%%%%%%%%%%
\newcommand{\Vector}[1]{\Matrix{#1}}
\newcommand*{\SET}[1]  {\ensuremath{\mathcal{#1}}}
\newcommand*{\MAT}[1]  {\ensuremath{\mathbf{#1}}}
\newcommand*{\VEC}[1]  {\ensuremath{\bm{#1}}}
\newcommand*{\CONST}[1]{\ensuremath{\mathit{#1}}}

%%%%%%%%%%%%%%%%%%%%%%%%%%%%%%%%%%%%%%%%%%%%%%%%%%%%%%%%%%%%%%%%%%%%%%%%%%%%%%%%
% Multicol Settings
%%%%%%%%%%%%%%%%%%%%%%%%%%%%%%%%%%%%%%%%%%%%%%%%%%%%%%%%%%%%%%%%%%%%%%%%%%%%%%%%
\setlength{\columnsep}{0.7em}
\setlength{\columnseprule}{0mm}

%%%%%%%%%%%%%%%%%%%%%%%%%%%%%%%%%%%%%%%%%%%%%%%%%%%%%%%%%%%%%%%%%%%%%%%%%%%%%%%%
% Save space in lists. Use this after the opening of the list
%%%%%%%%%%%%%%%%%%%%%%%%%%%%%%%%%%%%%%%%%%%%%%%%%%%%%%%%%%%%%%%%%%%%%%%%%%%%%%%%
\newcommand{\compresslist}{
	\setlength{\itemsep}{1pt}
	\setlength{\parskip}{0pt}
	\setlength{\parsep}{0pt}
}


%%%%%%%%%%%%%%%%%%%%%%%%%%%%%%%%%%%%%%%%%%%%%%%%%%%%%%%%%%%%%%%%%%%%%%%%%%%%%%
%%% Beginning of Poster
%%%%%%%%%%%%%%%%%%%%%%%%%%%%%%%%%%%%%%%%%%%%%%%%%%%%%%%%%%%%%%%%%%%%%%%%%%%%%%

\begin{document}
\typeout{Poster Starts}

\background{
	\begin{tikzpicture}[remember picture,overlay]
		\draw (current page.north west)+(-2em,-0em) node[anchor=north west]
			{\hspace{-2em}\includegraphics[height=1.1\textheight {silhouettes_background}};
	\end{tikzpicture}
}

\definecolor{silver}{cmyk}{0,0,0,0.3}
\definecolor{yellow}{cmyk}{0,0,0.9,0.0}
\definecolor{reddishyellow}{cmyk}{0,0.22,1.0,0.0}
\definecolor{black}{cmyk}{0,0,0.0,1.0}
\definecolor{darkYellow}{cmyk}{0,0,1.0,0.5}
\definecolor{darkSilver}{cmyk}{0,0,0,0.1}
\definecolor{lightyellow}{cmyk}{0,0,0.3,0.0}
\definecolor{lighteryellow}{cmyk}{0,0,0.1,0.0}
\definecolor{lighteryellow}{cmyk}{0,0,0.1,0.0}
\definecolor{lightestyellow}{cmyk}{0,0,0.05,0.0}
\definecolor{lightblue}{cmyk}{0.83,0.24,0,0.12}
\definecolor{darkblue}{cmyk}{1.0,0.90,0.13,0.61}

% Draw a video
\newlength{\FSZ}
\newcommand{\drawvideo}[3]{
	\noindent\pgfmathsetlength{\FSZ}{\linewidth/#2}
   	\begin{tikzpicture}[outer sep=0pt,inner sep=0pt,x=\FSZ,y=\FSZ]
   	\draw[color=lightblue!50!black] (0,0) node[outer sep=0pt,inner sep=0pt,text width=\linewidth,minimum height=0](video){\noindent#3};
   	\path[fill=lightblue!50!black,line width=0pt](video.north west)rectangle([yshift=\FSZ]video.north east) 
    \foreach \x in {1,2,...,#2} {
      {[rounded corners=0.6] ($(video.north west)+(-0.7,0.8)+(\x,0)$) rectangle +(0.4,-0.6)}
    };
   \path [fill=lightblue!50!black,line width=0pt]([yshift=-1\FSZ]video.south west)rectangle(video.south east) 
    \foreach \x in {1,2,...,#2} {
      {[rounded corners=0.6] ($(video.south west)+(-0.7,-0.2)+(\x,0)$) rectangle +(0.4,-0.6)}
    };
   \end{tikzpicture}
}


\begin{poster}{
  % Show grid to help with alignment
  grid=false,
  % Column spacing
  colspacing=1em,
  % Color style
  bgColorOne=white,
  bgColorTwo=white,
  borderColor=lightblue,
  headerColorOne=darkblue,
  headerColorTwo=lightblue,
  headerFontColor=white,
  boxColorOne=white,
  boxColorTwo=lightblue,
  % Format of textbox
  textborder=roundedleft,
  % Format of text header
  eyecatcher=false,
  headerborder=closed,
  headerheight=0.1\textheight,
  headershape=roundedright,
  headershade=shadelr,
  headerfont=\Large\bf\textsc,
  boxshade=plain,
  background=plain,
  linewidth=2pt
  }
  % Eye Catcher
  {}
  % Title
  {\bf\textsc{Using CHASTE (Oxford) to Explore Tissue Dynamics}}
  % Authors
  {\textsc{Dhananjay Bhaskar, Leah Keshet\hspace{2em}Department of Mathematics\hspace{2em}University of British Columbia}\vspace{0.2em}}
  % University logo
  {
  {\begin{minipage}{16em}
	\hfill
	\includegraphics[height=5em]{ChasteLogo}\hspace*{15pt}
	\includegraphics[height=5em]{pimslogo}\hspace*{15pt}
	\includegraphics[height=5em]{ubclogo}
  \end{minipage}}
  }

  \tikzstyle{light shaded}=[top color=baposterBGtwo!30!white,bottom color=baposterBGone!30!white,shading=axis,shading angle=30]

  % Width of left inset image
  \newlength{\leftimgwidth}
  \setlength{\leftimgwidth}{0.78em+8.0em}


%%%%%%%%%%%%%%%%%%%%%%%%%%%%%%%%%%%%%%%%%%%%%%%%%%%%%%%%%%%%%%%%%%%%%%%%%%%%%%
%%% Now define the boxes that make up the poster
%%%---------------------------------------------------------------------------
%%% Each box has a name and can be placed absolutely or relatively.
%%% The only inconvenience is that you can only specify a relative position 
%%% towards an already declared box. So if you have a box attached to the 
%%% bottom, one to the top and a third one which should be in between, you 
%%% have to specify the top and bottom boxes before you specify the middle 
%%% box.
%%%%%%%%%%%%%%%%%%%%%%%%%%%%%%%%%%%%%%%%%%%%%%%%%%%%%%%%%%%%%%%%%%%%%%%%%%%%%%
    
   
% A coloured circle useful as a bullet
\newcommand{\colouredcircle}{
      \tikz{
      	\useasboundingbox (-0.2em,-0.32em) rectangle(0.2em,0.32em); 
      	\draw[draw=black,fill=lightblue,line width=0.03em] (0,0) circle(0.18em);
      }
}


%%%%%%%%%%%%%%%%%%%%%%%%%%%%%%%%%%%%%%%%%%%%%%%%%%%%%%%%%%%%%%%%%%%%%%%%%%%%%%
\headerbox{CHASTE}{name=chasteintro,column=0,row=0}{
%%%%%%%%%%%%%%%%%%%%%%%%%%%%%%%%%%%%%%%%%%%%%%%%%%%%%%%%%%%%%%%%%%%%%%%%%%%%%%
	\textbf{Cancer, Heart and Soft Tissue Environment}
	\begin{itemize}[leftmargin=0.5cm,itemsep=2pt]
	\item[\colouredcircle] Extensible, general purpose open-source simulation framework: \textcolor{darkblue}{www.cs.ox.ac.uk/chaste/}
	\item[\colouredcircle] Applications in cardiac electrophysiology, systems biology and cell-based modeling
	\item[\colouredcircle] Research focus on exploring 2D tissue growth, rearrangement, morphogenesis
	\item[\colouredcircle] Useful for comparing distinct simulation paradigms
	\end{itemize}
}



%%%%%%%%%%%%%%%%%%%%%%%%%%%%%%%%%%%%%%%%%%%%%%%%%%%%%%%%%%%%%%%%%%%%%%%%%%%%%%
\headerbox{Developing Simulations}{name=simulation,column=0,below=chasteintro}{
%%%%%%%%%%%%%%%%%%%%%%%%%%%%%%%%%%%%%%%%%%%%%%%%%%%%%%%%%%%%%%%%%%%%%%%%%%%%%%
	Multiscale models simulate processes on two or more biological scales:\\
	\vspace{-1.2em}
	\begin{center}
		\includegraphics[scale=0.175]{scales_box}
	\end{center}
	\vspace{-1.4em}
	\hfill {\scriptsize \textcolor{silver}{Credit:\thinspace{BEeSy, Grenoble, France}}}
	\vspace{-1.6em}
    \begin{multicols}{2}
		\begin{enumerate}[leftmargin=0.5cm,itemsep=1.0pt]
			\item \textbf{Subcellular:} mitosis, cell death, cell cycle models
			\item \textbf{Cell:} inter - cellular forces, cell geometry
			\item \textbf{Tissue:} growth factors, diffusion, domain boundaries 
		\end{enumerate}
	\columnbreak
		\null \vfill
		\includegraphics[scale=0.18]{chastesim}  
		\vfill \null
	\end{multicols}  
	\vspace{-2.0em}
}


%%%%%%%%%%%%%%%%%%%%%%%%%%%%%%%%%%%%%%%%%%%%%%%%%%%%%%%%%%%%%%%%%%%%%%%%%%%%%%
\headerbox{References}{name=refs,column=0,below=simulation,above=bottom}{
%%%%%%%%%%%%%%%%%%%%%%%%%%%%%%%%%%%%%%%%%%%%%%%%%%%%%%%%%%%%%%%%%%%%%%%%%%%%%%
    \smaller
    \vspace{-0.4em}
    \bibliographystyle{ieee}
    \renewcommand{\section}[2]{\vskip 0.05em}
    
	\begin{thebibliography}{1}\itemsep=-0.01em
	
      \setlength{\baselineskip}{0.4em}
      
      \bibitem{fletcher:vertexdynamics}
     	A.~Fletcher, J.~Osborne, et al. 
		\newblock {I}mplementing {V}ertex {D}ynamics {M}odels of {C}ell {P}opulation in {B}iology within a {C}onsistent {C}omputational {F}ramework. 
      	\newblock {\em Progress in Biophysics and Molecular Biology, 113:299-326, 2013}
      	
      \bibitem{vossbohme:cpm}
	 	A.~Voss-B\"{o}hme.
	 	\newblock {M}ulti-{S}cale {M}odeling in {M}orphogenesis: A {C}ritical {A}nalysis of the {C}ellular {P}otts {M}odel.
	 	\newblock {\em PLoS ONE, 7(9): e42852, 2012}
	 	
	 \bibitem{chaste:opensourcelibrary}
	 	G.~Mirams, C.~Arthurs, et al.
	 	\newblock {C}haste: An {O}pen {S}ource C++ {L}ibrary for {C}omputational {P}hysiology and {B}iology.
	 	\newblock {\em PLoS Computational Biology, 9(3): e1002970, 2013}
	 	
	 \bibitem{pathmanathan:discretetissuemodel}
	  	P.~Pathmanathan, J.~Cooper, et al.
 		\newblock A {C}omputational {S}tudy of {D}iscrete {M}echanical {T}issue {M}odels.
 		\newblock {\em Physical Biology, 6(3), 2009}
      
      \bibitem{nagai:nagaihondamodel}
      	T.~Nagai and H.~Honda.
		\newblock A {D}ynamic {C}ell {M}odel for the {F}ormation of {E}pithelial {T}issues.
		\newblock {\em Philosophical Magazine, 81:699-719}

	\end{thebibliography}
}


%%%%%%%%%%%%%%%%%%%%%%%%%%%%%%%%%%%%%%%%%%%%%%%%%%%%%%%%%%%%%%%%%%%%%%%%%%%%%%
\headerbox{On-Lattice Models}{name=onlattice,column=1,row=0}{
%%%%%%%%%%%%%%%%%%%%%%%%%%%%%%%%%%%%%%%%%%%%%%%%%%%%%%%%%%%%%%%%%%%%%%%%%%%%%%
	\textbf{Cellular Automata:}\\
	\begin{minipage}{0.7\linewidth}
		\begin{itemize}[leftmargin=0.5cm]\compresslist
		\item[\colouredcircle] Fixed lattice, cells divide into neighbouring lattice sites
		\item[\colouredcircle] Simple implementation, computationally inexpensive
		\end{itemize}
	\end{minipage}
	\begin{minipage}{0.2\linewidth}
		\includegraphics[scale=0.30]{automata}  
	\end{minipage} 
	
	\textbf{Cellular Potts Model:}\\
	\begin{minipage}{0.7\linewidth}
		\vspace{0.3cm}
		\begin{itemize}[leftmargin=0.5cm]\compresslist
		\item[\colouredcircle] Cells are composed of collection of lattice sites
		\item[\colouredcircle] Cell shape can deform (non-viscous)
		\item[\colouredcircle] Sites are added or removed in Monte - Carlo updates to minimize free energy function:
		\end{itemize}
	\end{minipage}
	\begin{minipage}{0.2\linewidth}
		\includegraphics[scale=0.15]{potts}  
	\end{minipage}
	\begin{displaymath}
	\smaller
	H = \sum_{\text{cells:}w}\lambda[A_{w} - A_{\text{target}}]^2 + \sum_{\text{interfaces:$\{x,y\}$}}J(x,y) + H_{0}
	\end{displaymath}
	\small
	$A_{w}$ : area of cell $w$,
	$J$ : cell-cell adhesion energy
}
  
  
%%%%%%%%%%%%%%%%%%%%%%%%%%%%%%%%%%%%%%%%%%%%%%%%%%%%%%%%%%%%%%%%%%%%%%%%%%%%%%
\headerbox{Off-Lattice Models}{name=offlatice,column=1,below=onlattice,span=1,above=bottom}{
%%%%%%%%%%%%%%%%%%%%%%%%%%%%%%%%%%%%%%%%%%%%%%%%%%%%%%%%%%%%%%%%%%%%%%%%%%%%%%
	\textbf{Cell Centre Model:}\\
	\begin{minipage}{0.7\linewidth}
		\vspace{0.3cm}
		\begin{itemize}[leftmargin=0.5cm]\compresslist
		\item[\colouredcircle] Cells are points in space
		\item[\colouredcircle] Forces between centres modelled as overdamped springs:\\
		$\smaller \gamma \dfrac{dr_{i}}{dt} = \sum_{j} F_{ij} \dfrac{r_{i} - r_{j}}{\|r_{i} - r_{j}\|}$\\
		\item[\colouredcircle] No direct control over cell size and cell - cell adhesion
		\item[\colouredcircle] Cell connectivity:\\
			\smaller
			\begin{itemize}[leftmargin=0.3cm]\compresslist
			\vspace{-0.4cm}
			\item Overlapping Spheres Model
			\item Voronoi Tessellation
			\end{itemize}
		\end{itemize}
	\end{minipage}
	\begin{minipage}{0.2\linewidth}
		\includegraphics[scale=0.15]{cellcentre}  
	\end{minipage}
	
	\vspace{0.3cm}
	
	\textbf{Vertex Dynamics Model:}\\
	\begin{minipage}{0.7\linewidth}
		\vspace{0.3cm}
		\begin{itemize}[leftmargin=0.5cm]\compresslist
		\item[\colouredcircle] Cells represented as polygons whose vertices are free to move
		\item[\colouredcircle] Good model for epithelia that exhibit regular shape
		\item[\colouredcircle] Modelling force on each vertex:
			\smaller
			\begin{itemize}[leftmargin=0.3cm]\compresslist
			\item Explicit: Weliky \& Oster (1990)
			\item Potential gradient: Nagai \& Honda (2001)
			\end{itemize}
		\end{itemize}
	\end{minipage}
	\begin{minipage}{0.2\linewidth}
		\includegraphics[scale=0.15]{vertexdynamics}  
	\end{minipage}  
	\vspace{-0.2cm}
	\begin{eqnarray*}
	\smaller
	\gamma\dfrac{dr_{i}}{dt} = - \nabla_{i} U &=& -\nabla_{i}\sum_{\text{cells:}w}(\lambda(A_{w} - A_{\text{target}})^2 \\[-0.08cm]
	&+& \beta(C_{w} - C_{\text{target}})^2 + U_{\text{adhesion}})
	\end{eqnarray*}
}
  
  
%%%%%%%%%%%%%%%%%%%%%%%%%%%%%%%%%%%%%%%%%%%%%%%%%%%%%%%%%%%%%%%%%%%%%%%%%%%%%%
\headerbox{Tissue Dynamics}{name=results,column=2,span=2,row=0}{
%%%%%%%%%%%%%%%%%%%%%%%%%%%%%%%%%%%%%%%%%%%%%%%%%%%%%%%%%%%%%%%%%%%%%%%%%%%%%%
	\vspace{0.2cm}

	A comparison of cell centre and vertex dynamics models with stochastic, area dependent cell division. We observe contact inhibition due to pressure exerted by neighbouring cells in a confined domain (non-dividing cells depicted by blue labelling):\\
   
   \drawvideo{5}{40}{
      	\includegraphics[width=0.18\linewidth]{contactin1}
        \includegraphics[width=0.18\linewidth]{contactin2}
        \includegraphics[width=0.18\linewidth]{contactin3}
        \includegraphics[width=0.18\linewidth]{contactin4}
        \includegraphics[width=0.18\linewidth]{contactin5}
        
        \includegraphics[width=0.18\linewidth]{vertexin1}
        \includegraphics[width=0.18\linewidth]{vertexin2}
        \includegraphics[width=0.18\linewidth]{vertexin3}
        \includegraphics[width=0.18\linewidth]{vertexin4}
        \includegraphics[width=0.18\linewidth]{vertexin5}
  }
  \\[-0.6cm]
   
  Progeny of an uninhibited stem cell (turquoise) dominate the monolayer in the absence of a mechanism to arrest cell division at high density:\\
   
   \drawvideo{5}{40}{
      	\includegraphics[width=0.18\linewidth]{monoclonal1}
        \includegraphics[width=0.18\linewidth]{monoclonal2}
        \includegraphics[width=0.18\linewidth]{monoclonal3}
        \includegraphics[width=0.18\linewidth]{monoclonal4}
        \includegraphics[width=0.18\linewidth]{monoclonal5}
  }
  \\[-0.6cm]
  
  Differential cell adhesion drives cell sorting by controlling cell-cell contact formation. In the following simulation based on vertex dynamics model, labelled cells (in blue) tend to clump together to minimize adhesion energy $(U_{\text{adhesion}})$ of the system:\\
  
  \drawvideo{5}{40}{
      	\includegraphics[width=0.18\linewidth]{vertexad1}
        \includegraphics[width=0.18\linewidth]{vertexad2}
        \includegraphics[width=0.18\linewidth]{vertexad3}
        \includegraphics[width=0.18\linewidth]{vertexad4}
        \includegraphics[width=0.18\linewidth]{vertexad5}
  }	
}
        %\includegraphics[width=0.18\linewidth]{potts1}
        %\includegraphics[width=0.18\linewidth]{potts2}
        %\includegraphics[width=0.18\linewidth]{potts3}
        %\includegraphics[width=0.18\linewidth]{potts4}
        %\includegraphics[width=0.18\linewidth]{potts5}
  
\end{poster}

\end{document}
