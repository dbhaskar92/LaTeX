\documentclass[landscape,a0paper,fontscale=0.292]{baposter}

\usepackage[vlined]{algorithm2e}
\usepackage{times}
\usepackage{calc}
\usepackage{url}
\usepackage{graphicx}
\usepackage{amsmath}
\usepackage{amssymb}
\usepackage{relsize}
\usepackage{multirow}
\usepackage{booktabs}

\usepackage{graphicx}
\usepackage{enumitem}
\usepackage{multicol}
\usepackage[T1]{fontenc}
\usepackage{ae}

\graphicspath{{images/}}

 %%%%%%%%%%%%%%%%%%%%%%%%%%%%%%%%%%%%%%%%%%%%%%%%%%%%%%%%%%%%%%%%%%%%%%%%%%%%%%%%
 %%%% Some math symbols used in the text
 %%%%%%%%%%%%%%%%%%%%%%%%%%%%%%%%%%%%%%%%%%%%%%%%%%%%%%%%%%%%%%%%%%%%%%%%%%%%%%%%
 % Format 
 \newcommand{\RotUP}[1]{\begin{sideways}#1\end{sideways}}


 %%%%%%%%%%%%%%%%%%%%%%%%%%%%%%%%%%%%%%%%%%%%%%%%%%%%%%%%%%%%%%%%%%%%%%%%%%%%%%%%
 % Multicol Settings
 %%%%%%%%%%%%%%%%%%%%%%%%%%%%%%%%%%%%%%%%%%%%%%%%%%%%%%%%%%%%%%%%%%%%%%%%%%%%%%%%
 \setlength{\columnsep}{0.7em}
 \setlength{\columnseprule}{0mm}


 %%%%%%%%%%%%%%%%%%%%%%%%%%%%%%%%%%%%%%%%%%%%%%%%%%%%%%%%%%%%%%%%%%%%%%%%%%%%%%%%
 % Save space in lists. Use this after the opening of the list
 %%%%%%%%%%%%%%%%%%%%%%%%%%%%%%%%%%%%%%%%%%%%%%%%%%%%%%%%%%%%%%%%%%%%%%%%%%%%%%%%
 \newcommand{\compresslist}{%
 \setlength{\itemsep}{1pt}%
 \setlength{\parskip}{0pt}%
 \setlength{\parsep}{0pt}%
 }


 %%%%%%%%%%%%%%%%%%%%%%%%%%%%%%%%%%%%%%%%%%%%%%%%%%%%%%%%%%%%%%%%%%%%%%%%%%%%%%
 % Formating
 \newcommand{\Matrix}[1]{\begin{bmatrix} #1 \end{bmatrix}}
 \newcommand{\Vector}[1]{\begin{pmatrix} #1 \end{pmatrix}}

 \newcommand*{\norm}[1]{\mathopen\| #1 \mathclose\|}% use instead of $\|x\|$
 \newcommand*{\abs}[1]{\mathopen| #1 \mathclose|}% use instead of $\|x\|$
 \newcommand*{\normLR}[1]{\left\| #1 \right\|}% use instead of $\|x\|$

 \newcommand*{\SET}[1]  {\ensuremath{\mathcal{#1}}}
 \newcommand*{\FUN}[1]  {\ensuremath{\mathcal{#1}}}
 \newcommand*{\MAT}[1]  {\ensuremath{\boldsymbol{#1}}}
 \newcommand*{\VEC}[1]  {\ensuremath{\boldsymbol{#1}}}
 \newcommand*{\CONST}[1]{\ensuremath{\mathit{#1}}}

 \DeclareMathOperator*{\argmax}{arg\,max}
 \DeclareMathOperator*{\diag}{diag}
 \DeclareMathOperator*{\argmin}{arg\,min}
 \DeclareMathOperator*{\vectorize}{vec}
 \DeclareMathOperator*{\reshape}{reshape}

 %-----------------------------------------------------------------------------
 % Differentiation
 \newcommand*{\Nabla}[1]{\nabla_{\!#1}}

 \renewcommand*{\d}{\mathrm{d}}
 \newcommand*{\dd}{\partial}

 \newcommand*{\At}[2]{\ensuremath{\left.#1\right|_{#2}}}
 \newcommand*{\AtZero}[1]{\At{#1}{\pp=\VEC 0}}

 \newcommand*{\diffp}[2]{\ensuremath{\frac{\dd #1}{\dd #2}}}
 \newcommand*{\diffpp}[3]{\ensuremath{\frac{\dd^2 #1}{\dd #2 \dd #3}}}
 \newcommand*{\diffppp}[4]{\ensuremath{\frac{\dd^3 #1}{\dd #2 \dd #3 \dd #4}}}
 \newcommand*{\difff}[2]{\ensuremath{\frac{\d #1}{\d #2}}}
 \newcommand*{\diffff}[3]{\ensuremath{\frac{\d^2 #1}{\d #2 \d #3}}}
 \newcommand*{\difffp}[3]{\ensuremath{\frac{\dd\d #1}{\d #2 \dd #3}}}
 \newcommand*{\difffpp}[4]{\ensuremath{\frac{\dd^2\d #1}{\d #2 \dd #3 \dd #4}}}

 \newcommand*{\diffpAtZero}[2]{\ensuremath{\AtZero{\diffp{#1}{#2}}}}
 \newcommand*{\diffppAtZero}[3]{\ensuremath{\AtZero{\diffpp{#1}{#2}{#3}}}}
 \newcommand*{\difffAt}[3]{\ensuremath{\At{\difff{#1}{#2}}{#3}}}
 \newcommand*{\difffAtZero}[2]{\ensuremath{\AtZero{\difff{#1}{#2}}}}
 \newcommand*{\difffpAtZero}[3]{\ensuremath{\AtZero{\difffp{#1}{#2}{#3}}}}
 \newcommand*{\difffppAtZero}[4]{\ensuremath{\AtZero{\difffpp{#1}{#2}{#3}{#4}}}}

 %-----------------------------------------------------------------------------
 % Defined
 % How should the defined operator look like (:= or ^= ==)
 % (I want back my :=, it is so much better than ^= because (1) it has a
 % direction and (2) everyone here uses it.)
 %
 % Use :=
 %\newcommand*{\defined}{\ensuremath{\mathrel{\mathop{:}}=}}
 %\newcommand*{\definedRight}{\ensuremath{=\mathrel{\mathop{:}}}}
 % Use ^=
 \newcommand*{\defined}{\ensuremath{\triangleq}}
 \newcommand*{\definedRight}{\ensuremath{\triangleq}}
 % Use = with three bars
 %\newcommand*{\defined}{\ensuremath{?}}
 %\newcommand*{\definedRight}{\ensuremath{?}}

 %%%%%%%%%%%%%%%%%%%%%%%%%%%%%%%%%%%%%%%%%%%%%%%%%%%%%%%%%%%%%%%%%%%%%%%%%%%%%%
 % Symbols used in the paper

 %-----------------------------------------------------------------------------
 % The Methods
 \newcommand*{\ICIA}{\emph{ICIA}}
 \newcommand*{\CoDe}{\emph{CoDe}}
 \newcommand*{\LinCoDe}{\emph{LinCoDe}}
 \newcommand*{\CoNe}{\emph{CoNe}}
 \newcommand*{\CoLiNe}{\emph{CoLiNe}}
 \newcommand*{\LinCoLiNe}{\emph{LinCoLiNe}}

 % inter eye distance
 \newcommand*{\ied}{IED}

 %-----------------------------------------------------------------------------
 % Koerper
 %%\newcommand*{\RR}{\mathbb{R}}
 %\newcommand*{\RR}{{I\hspace{-3.5pt}R}}
 %\newcommand*{\RR}{{\mathrm{I\hspace{-2.7pt}R}}}

 \font\dsfnt=dsrom12

 \DeclareSymbolFont{nark}{U}{dsrom}{m}{n}
 \DeclareMathSymbol{\NN}{\dsfnt}{nark}{`N}
 \DeclareMathSymbol{\RR}{\dsfnt}{nark}{`R}
 \DeclareMathSymbol{\ZZ}{\dsfnt}{nark}{`Z}

 %-----------------------------------------------------------------------------
 % Domains
 \newcommand*{\D}{\mathcal{D}}
 \newcommand*{\I}{\mathcal{I}}

 %-----------------------------------------------------------------------------
 % Texture coordinates
 \newcommand*{\rr}{\VEC{r}}

 %-----------------------------------------------------------------------------
 % Parameters
 \newcommand*{\pt}{\VEC{\tau}}
 \newcommand*{\pr}{\VEC{\rho}}
 \newcommand*{\pp}{\VEC{p}}
 \newcommand*{\qq}{\VEC{q}}
 \newcommand*{\xx}{\VEC{x}}
 \newcommand*{\deltaq}{\Delta \qq}
 \newcommand*{\deltap}{\Delta \pp}
 \newcommand*{\zz}{\VEC{z}}
 \newcommand*{\pa}{\VEC{\alpha}}
 \newcommand*{\qa}{\VEC{\alpha}}
 \newcommand*{\pb}{\VEC{\beta}}

 %-----------------------------------------------------------------------------
 % Optimal appearance parameters
 \newcommand*{\pbh}[1]{\ensuremath{\hat{\pb}({#1})}}

 %-----------------------------------------------------------------------------
 % Warp basis
 \newcommand*{\M}[1]{\ensuremath{M({#1})}}
 \newcommand*{\LL}[1]{\ensuremath{L({#1})}}

 %-----------------------------------------------------------------------------
 % Matrices of the texture model
 \newcommand*{\AM}[1]{\ensuremath{\Lambda(#1)}}               	% Lambda(beta) 
 \newcommand*{\AMr}[2]{\ensuremath{\Lambda(#1; #2)}}        	% Lambda(r, beta)

 \newcommand*{\As}{A}         % Continuous Basis symbol
 \newcommand*{\afs}{a}        % Continuous mean symbol
 \newcommand*{\A}[1]{\As(#1)}         % Continuous Basis
 \newcommand*{\af}[1]{\afs(#1)}        % Continuous mean


 %-----------------------------------------------------------------------------
 % Matrices of the shape model
 \newcommand*{\MU}{\VEC{\mu}}
 \newcommand*{\MM}{\MAT{M}}

 %-----------------------------------------------------------------------------
 %% The project out matrix and operator
 \newcommand*{\INT}{\MAT{P}}
 \newcommand*{\INTf}{P}

 %-----------------------------------------------------------------------------
 % The identity matrix
 \newcommand*{\EYEtwo}{\Matrix{1 & 0\\0&1}}
 \newcommand*{\EYE}{\MAT E}
 \newcommand*{\EYEf}{E}

 % Wether to use subscripts or brackets for some function arguments
 % can be decided by commenting out the corresponding functions underneath
 %-----------------------------------------------------------------------------
 % Mapping
 \newcommand*{\Cs}[1]{\ensuremath{C^{#1}}} % C symbol
 \newcommand*{\C}[2]{\ensuremath{C^{#1}(#2)}} % Use C with brackets

 %-----------------------------------------------------------------------------
 % Objective function
 \newcommand*{\Fs}{\ensuremath{F}}              	% F symbol
 \newcommand*{\F}[1]{\ensuremath{\Fs(#1)}}       	% Use F with brackets    F(q)

 %-----------------------------------------------------------------------------
 % Approximated objective functions
 \newcommand*{\FFs}{\tilde{F}}                     % ~F symbol
 \newcommand*{\FF}[1]{\ensuremath{\FFs(#1)}}       % Use ~F with brackets    F(q)

 %-----------------------------------------------------------------------------
 % residual function
 \newcommand*{\es}{\ensuremath{f}}              	% R symbol

 \newcommand*{\e}[1]{\ensuremath{\es(#1)}}         	% R(q)
 \newcommand*{\er}[2]{\ensuremath{\es(#1; #2)}}    	% R(r; q)

 %-----------------------------------------------------------------------------
 % Approximated residual functions
 \newcommand*{\ees}{\tilde{f}}                       	% ~R symbol
 \newcommand*{\ee}[1]{\ensuremath{\ees(#1)}}       		% ~R(q)
 \newcommand*{\eer}[2]{\ensuremath{\ees(#2; #1)}}  		% ~R(r; q)

 %-----------------------------------------------------------------------------
 % Warps
 \newcommand*{\Vs}{\ensuremath{V}}
 \newcommand*{\VLins}{\ensuremath{\Vs^{\text{Ortho}}}}
 \newcommand{\VModels}{\ensuremath{\Vs^{\text{Model}}}}
 \newcommand*{\Ws}{\ensuremath{W}}

 \newcommand{\V}[1]{\ensuremath{\Vs(#1)}}
 \newcommand{\VModel}[1]{\ensuremath{\VModels(#1)}}
 \newcommand{\Vr}[2]{\ensuremath{\Vs(#1; #2)}}
 \newcommand{\VInvr}[2]{\ensuremath{\Vs^{-1}(#1; #2)}}
 \newcommand{\VrLin}[2]{\ensuremath{\VLins(#1; #2)}}
 \newcommand{\W}[1]{\ensuremath{\Ws(#1)}}
 \newcommand{\Winv}[1]{\ensuremath{\Ws^{-1}(#1)}}
 \newcommand{\Wr}[2]{\ensuremath{\Ws(#1; #2)}}


%%%%%%%%%%%%%%%%%%%%%%%%%%%%%%%%%%%%%%%%%%%%%%%%%%%%%%%%%%%%%%%%%%%%%%%%%%%%%
%% Begin of Document
%%%%%%%%%%%%%%%%%%%%%%%%%%%%%%%%%%%%%%%%%%%%%%%%%%%%%%%%%%%%%%%%%%%%%%%%%%%%%
\begin{document}
%%%%%%%%%%%%%%%%%%%%%%%%%%%%%%%%%%%%%%%%%%%%%%%%%%%%%%%%%%%%%%%%%%%%%%%%%%%%%
%% Here starts the poster
%%---------------------------------------------------------------------------
%% Format it to your taste with the options
%%%%%%%%%%%%%%%%%%%%%%%%%%%%%%%%%%%%%%%%%%%%%%%%%%%%%%%%%%%%%%%%%%%%%%%%%%%%%
\begin{poster}{
 % Show grid to help with alignment
 grid=false,
 % Column spacing
 colspacing=0.7em,
 % Color style
 headerColorOne=cyan!20!white!90!black,
 borderColor=cyan!30!white!90!black,
 % Format of textbox
 textborder=faded,
 % Format of text header
 headerborder=open,
 headershape=roundedright,
 headershade=plain,
 background=none,
 bgColorOne=cyan!10!white,
 headerheight=0.12\textheight}
 % Eye Catcher
 {
      \includegraphics[height=0.10\textheight]{ubclogo}
 }
 % Title
 {\sc\Huge Invasion Fronts and Tissue Dynamics}
 % Authors
 {Dhananjay Bhaskar and Leah Edelstein-Keshet\\[1em]
 {\texttt{dbhaskar92@gmail.com, keshet@math.ubc.ca}}}
 % University logo
 {
  \begin{tabular}{r}
    \includegraphics[height=0.12\textheight]{murc}
  \end{tabular}
 }

%%%%%%%%%%%%%%%%%%%%%%%%%%%%%%%%%%%%%%%%%%%%%%%%%%%%%%%%%%%%%%%%%%%%%%%%%%%%%%
%%% Now define the boxes that make up the poster
%%%---------------------------------------------------------------------------
%%% Each box has a name and can be placed absolutely or relatively.
%%% The only inconvenience is that you can only specify a relative position 
%%% towards an already declared box. So if you have a box attached to the 
%%% bottom, one to the top and a third one which should be inbetween, you 
%%% have to specify the top and bottom boxes before you specify the middle 
%%% box.
%%%%%%%%%%%%%%%%%%%%%%%%%%%%%%%%%%%%%%%%%%%%%%%%%%%%%%%%%%%%%%%%%%%%%%%%%%%%%%

%%%%%%%%%%%%%%%%%%%%%%%%%%%%%%%%%%%%%%%%%%%%%%%%%%%%%%%%%%%%%%%%%%%%%%%%%%%%%%
\headerbox{Introduction}{name=contribution,column=0,row=0,span=2}{
%%%%%%%%%%%%%%%%%%%%%%%%%%%%%%%%%%%%%%%%%%%%%%%%%%%%%%%%%%%%%%%%%%%%%%%%%%%%%%
Solid tumors are thought to arise from single, motile cells which have experienced loss of cell polarity and cell-cell adhesion; a process known as \textbf{epithelial-to-mesenchymal transformation}. However recent research suggests that \textbf{collective invasion}, wherein cells maintain connections with their neighbors and move together through the extracellular matrix, plays an important role in driving human breast cancer malignancy. We have developed an in silico model that is calibrated to simulate the experimental setup used to study breast cancer invasion.\\ 

Based on this work, we are able to suggest new experimental setups for further studies in laboratory. We can understand the emergence of different patterns in scratch wound healing assays that are widely used to study \textbf{mammalian cell migration} and \textbf{tumor cell motility}.
}

  
%%%%%%%%%%%%%%%%%%%%%%%%%%%%%%%%%%%%%%%%%%%%%%%%%%%%%%%%%%%%%%%%%%%%%%%%%%%%%%
\headerbox{Multiscale Modeling}{name=abstract,column=0,below=contribution}{
%%%%%%%%%%%%%%%%%%%%%%%%%%%%%%%%%%%%%%%%%%%%%%%%%%%%%%%%%%%%%%%%%%%%%%%%%%%%%%
Multiscale models simulate processes on two or more biological scales:\\
\vspace{-1.2em}
\begin{center}
	\includegraphics[scale=0.22]{scales}
\end{center}
\vspace{-1.4em}
\begin{multicols}{2}
	\begin{enumerate}[leftmargin=0.4cm,itemsep=1.0pt]
		\item \textbf{Subcellular:} mitosis, cell death, cell cycle models
		\item \textbf{Cell:} inter - cellular forces, cell geometry, contact inhibition
		\item \textbf{Tissue:} growth factors, diffusion, domain boundaries 
	\end{enumerate}
\columnbreak
	\null \vfill
	\includegraphics[scale=0.18]{pottsim}  
	\vfill \null
\end{multicols}	
}


%%%%%%%%%%%%%%%%%%%%%%%%%%%%%%%%%%%%%%%%%%%%%%%%%%%%%%%%%%%%%%%%%%%%%%%%%%%%%%
\headerbox{Cellular Potts Model}{name=lowrestracking,column=1,span=1,below=contribution,above=bottom}{
%%%%%%%%%%%%%%%%%%%%%%%%%%%%%%%%%%%%%%%%%%%%%%%%%%%%%%%%%%%%%%%%%%%%%%%%%%%%%%
\begin{minipage}{0.7\linewidth}
	\begin{itemize}[leftmargin=*]\compresslist
	\item Cells are composed of collection of lattice sites
	\item Cell shape can deform (non-viscous)
	\item Sites are added or removed in Monte - Carlo updates to minimize free energy function:
	\end{itemize}
\end{minipage}
\begin{minipage}{0.2\linewidth}
	\includegraphics[scale=0.15]{potts}  
\end{minipage}
\begin{displaymath}
\smaller
H = \sum_{\text{cells:}w}\lambda[A_{w} - A_{\text{target}}]^2 + \sum_{\text{faces:$\{x,y\}$}}J(x,y) + H_{0}
\end{displaymath}
\small
$A_{w}$ : area of cell $w$,
$J$ : cell-cell adhesion energy

\begin{center}
\textbf{Initial Lattice Configuration}
\end{center}
\begin{center}
\includegraphics[scale=0.15]{latticecpm} 
\end{center}
\vspace{-0.1cm}
Total Number of Cells: 400\\
Number of MCFpIRES Cells (Labelled Green): 320\\
Number of MCFmPodo Cells (Labelled Red): 80\\
Tissue Width: 40 cells = 200 lattice sites\\
Tissue Height: 10 cells = 50 lattice sites
\begin{center}
\textbf{Cell Growth and Contact Inhibition}
\end{center}
\vspace{-0.1cm}
At each iteration, ratio of surface area of cell in contact with ECM and the total surface area is calculated. The rate of growth is determined by the sigmoidal function:
\begin{equation}
\frac{0.01}{(1 + e^{-2 \times \mbox{surface area ratio}})} \nonumber
\end{equation}
Change in volume of any cell is typically between 0.005 and 0.008 units per MCS.
}


%%%%%%%%%%%%%%%%%%%%%%%%%%%%%%%%%%%%%%%%%%%%%%%%%%%%%%%%%%%%%%%%%%%%%%%%%%%%%%
\headerbox{References}{name=references,column=0,above=bottom}{
%%%%%%%%%%%%%%%%%%%%%%%%%%%%%%%%%%%%%%%%%%%%%%%%%%%%%%%%%%%%%%%%%%%%%%%%%%%%%%
\smaller
\vspace{-0.4em}
\bibliographystyle{ieee}
\renewcommand{\section}[2]{\vskip 0.05em}

\begin{thebibliography}{1}\itemsep=-0.01em

\setlength{\baselineskip}{0.4em}

\bibitem{berrier:cellmatrix}
A.~Berrier and K.~Yamada.
\newblock {C}ell-Matrix {A}dhesion.
\newblock {\em J. Cell. Physiol., 213:565-573, 2007}

\bibitem{vossbohme:cpm}
A.~Voss-B\"{o}hme.
\newblock {M}ulti-{S}cale {M}odeling in {M}orphogenesis: A {C}ritical {A}nalysis of the {C}ellular {P}otts {M}odel.
\newblock {\em PLoS ONE, 7(9): e42852, 2012}

\bibitem{scianna:hybridmodel}
M.~Scianna and L.~Preziosi.
\newblock A {H}ybrid {M}odel Describing Different {M}orphologies of {T}umor {I}nvasion {F}ronts.
\newblock {\em Math. Model. Nat. Phenom., 7(1):78-104, 2012}

\bibitem{swat:compucell}
M.~Swat, G.~Thomas, J.~Belmonte, et al.
\newblock {M}ulti-Scale {M}odeling of {T}issues Using {C}ompuCell3D.
\newblock {\em Methods in Cell Biology, 110:325-366, 2012}

\bibitem{pathmanathan:discretetissuemodel}
P.~Pathmanathan, J.~Cooper, et al.
\newblock A {C}omputational {S}tudy of {D}iscrete {M}echanical {T}issue {M}odels.
\newblock {\em Physical Biology, 6(3), 2009}

\end{thebibliography}
}


%%%%%%%%%%%%%%%%%%%%%%%%%%%%%%%%%%%%%%%%%%%%%%%%%%%%%%%%%%%%%%%%%%%%%%%%%%%%%%
\headerbox{CompuCell3D Simulation Results}{name=speed,column=2,row=0,span=2}{
%%%%%%%%%%%%%%%%%%%%%%%%%%%%%%%%%%%%%%%%%%%%%%%%%%%%%%%%%%%%%%%%%%%%%%%%%%%%%%
   
\begin{multicols}{2}
	\includegraphics[scale=0.28]{res1}
	\columnbreak
	\includegraphics[scale=0.28]{res3}
\end{multicols}

\vspace{-0.05cm}

\includegraphics[scale=0.38]{res4}
\hspace{0.1cm}
\includegraphics[scale=0.35]{res21}

\textbf{Prelimnary Results:}
\begin{itemize}[leftmargin=*]\compresslist
\item Increasing the external force results in more cells being pulled into the ECM.
\item Cells situated immediately beneath the top layer are pulled as a result of cell-cell adhesion.
\item As cell-matrix adhesion increases, cells disperse into the ECM and collective migration is not observed.
\item Increasing the maximum growth rate increases supply of cells which fill gaps or tears in the tissue.
\end{itemize}

}

\end{poster}
\end{document}
