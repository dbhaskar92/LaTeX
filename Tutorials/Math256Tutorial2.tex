\documentclass[11pt]{article}

\usepackage{amsfonts}
\usepackage{amsmath}
\usepackage{amssymb}
\usepackage{graphicx}
\usepackage[left=2cm,top=1cm,right=2cm,foot=0cm,bottom=1cm]{geometry}
\usepackage[shortlabels]{enumitem}
\usepackage{color}
\usepackage{tikz}
\usepackage{arrayjob}

\newcommand{\comment}[1]{}
\newcommand{\blue}[1]{\textcolor{blue}{\Large \hspace{0.5cm} #1}}
\newcounter{choiceCtr}[enumi]
\newcommand{\choice}[1]{\addtocounter{choiceCtr}{1} \hspace{1.5cm} (\alph{choiceCtr}) #1}

\newcounter{choiceCtr2}[enumi]
\newcommand{\choiceroman}[1]{\addtocounter{choiceCtr2}{1} \hspace{0.2cm} (\roman{choiceCtr2}) #1}

\newcommand{\answerbox}[4]{%
\raisebox{-#3}{\framebox[#1][l]{\vphantom{\rule{0.01cm}{#2}} \raisebox{#3}{\blue{#4}}}} }
\renewcommand{\arraystretch}{1.1}
\setlength\parindent{0in}

\newcommand{\examdate}{January 18, 2016}
\newcommand{\course}{MATH 256}
\newcommand{\event}{Tutorial 2 Worksheet}

\newcounter{markcounter}
\newcommand{\marklabel}{\addtocounter{markcounter}{1} \arabic{markcounter}}
\newcommand{\markref}{\arabic{markcounter}}

\newcommand\blfootnote[1]{
  \begingroup
  \renewcommand\thefootnote{}\footnote{#1}
  \addtocounter{footnote}{-1}
  \endgroup
}

%% USAGE: \marklabel \blfootnote{\markref: \answerbox{1cm}{0.8cm}{0.4cm}{}}

%%
%% Author: Dhananjay Bhaskar
%% Last modified: January 15, 2016
%%

\begin{document}

% Title
\begin{center}
\fbox{\fbox{\parbox{5.5in}{\centering\Large \textbf{\course\:\event}\\
\examdate}}}
\end{center}

\vspace{0.5cm}
\makebox[\textwidth]{\Large Name (please print) :\enspace\hrulefill}\medskip
\vspace{0.5cm}
\makebox[\textwidth]{\Large Student Number :\enspace\hrulefill}

% Question 1

\subsection*{Question 1}

Are the following functions linearly independent or linearly dependent? Justify your claim. \\
\textbf{If they are independent, use the Wronskian to show that.} \\
\textbf{If they are not independent, give nonzero values of $c_1$, $c_2$, and $c_3$ such that:} \\
$c_1 f(x) + c_2 g(x) + c_3 h(x) = 0$.

\begin{enumerate}[(a)]
  \item f(x) = $x$, g(x) = $x^2$ and h(x) = $x^3$
  \vspace{7cm}
  \item f(x) = log($x$), g(x) = log($x^2$) and h(x) = log($x^3$)
\end{enumerate}

\newpage
\vspace{1cm}

% Question 2

\subsection*{Question 2}
An atom of a radioactive substance typically decays into an atom of some other radioactive substance with its own decay constant. Suppose we have $50$ grams of pure Uranium-238 with a decay constant $m$. Each atom of Uranium-238 decays to a single atom of Thorium-234 which has a decay constant $n$. We have no Thorium-234 initially.

\begin{enumerate}[(a)]
  \item Write down the differential equations and initial conditions for the radioactive decay of Uranium-238 ($U(t)$) and production and decay of Thorium-234 ($T(t)$).
  \vspace{3cm}
  \item Solve the differential equations and initial conditions to find the amount of Thorium-234 after time $t$ in terms of $m$ and $n$. You should find that there are two cases that must be treated separately, $m \ne n$ and $m = n$.
  \vspace{7.5cm}
  \item \textbf{(Optional)} Notice that the solution in the case of $n = m$ ($T_{n = m}(t)$) does not have the same form as the solution in the case of $n \ne m$ ($T_{n \ne m}(t)$). Show that as $m$ approaches $n$, the function $T_{n \ne m}(t)$ approaches $T_{n = m}(t)$.
\end{enumerate}


\end{document}
